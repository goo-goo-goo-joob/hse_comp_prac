%!TEX TS-program = xelatex

% Шаблон документа LaTeX создан в 2018 году
% Алексеем Подчезерцевым
% В качестве исходных использованы шаблоны
% 	Данилом Фёдоровых (danil@fedorovykh.ru) 
%		https://www.writelatex.com/coursera/latex/5.2.2
%	LaTeX-шаблон для русской кандидатской диссертации и её автореферата.
%		https://github.com/AndreyAkinshin/Russian-Phd-LaTeX-Dissertation-Template

\documentclass[a4paper,14pt]{article}

\input{data/preambular.tex}
\begin{document} % конец преамбулы, начало документа
\input{data/title.tex}
\setcounter{page}{2} % нумерация

\renewcommand\contentsname{\centering {\normalsize Содержание}}
\tableofcontents
\newpage

\section*{Постановка задачи}
\addcontentsline{toc}{section}{Постановка задачи}

Разработать шаблон класса структуры данных, включая «джентльменский набор» операций: конструктор «пустой» структуры данных, деструктор, операции добавления, включения и исключения по логическому номеру, сортировки и включения с сохранением порядка при наличии стандартным образом переопределенной операции сравнения объектов класса – параметра шаблона. Разработать также методы сохранения и загрузки структуры данных из стандартного текстового (или двоичного) потока при
наличии переопределенных операций \textit{<<} и \textit{>>} для потока и объектов хранимого класса – параметра шаблона. Загрузку структуры данных из стандартного потока производить в создаваемые для этой цели динамические объекты. Структуру данных выбрать в соответствии с вариантом 4: Циклическая очередь, представленная динамическим массивом указателей на хранимые объекты. Размерность очереди – параметр конструктора.

\newpage

\section{Основная часть}
\subsection{Общая идея решения задачи}
Для решения задачи был разработан шаблонный класс \textit{CircleQueue} со всеми необходимыми полями и методами. Также было использовано динамическое выделение памяти, перегрузка дружественных операторов ввода и вывода, и организован механизм генерирования исключений.
\subsection{Структура и принципы действия}
В лабораторной работе №3 был разработан класс, содержащий следующие поля: поле \textit{size} хранит текущий размер очереди, поле \textit{cap} содержит информацию о вместимости, \textit{head} обозначает позицию для записи следующего элемента, \textit{tail} содержит позицию начала очереди, поле \textit{values} содержит указатель на массив указателей на хранимые объекты, \textit{sorted} содержит флаг о том, отсортирована или нет очередь.

В соответствии с заданием был добавлен конструктор с параметром, выделяющий память для массива указателей заданной длины, деструктор.

Операция \textit{pushBack} реализует добавление элемента в конец очереди: динамически создается копия этого элемента и добавляется на первую свободную позицию. Если свободных мест нет, то генерируется исключение.

Операция вставки элемента на заданную позицию \textit{insert} сдвигает элементы массива на одну позицию назад и на освободившееся место добавляет новый элемент. Исключение генерируется  в случае, когда очередь уже полностью заполнена или позиция вставки больше, чем текущий размер очереди.

Метод удаления элемента по заданному индексу \textit{pop} проверяет на наличие элемента с нужным индексом в массиве, иначе генерирует исключение. В методе происходит сдвиг элементов с нужного индекса на одну позицию вперед. Возвращается удаленный элемент.

Сортировка элементов очереди \textit{sort} происходит методом сортировки пузырьком. При этом для элементов очереди должна быть переопределена операция сравнения.

Операция \textit{PushSort} реализует включение с сохранением порядка: если очередь была отсортирована, то элемент вставляется на место так, чтобы сортировка сохранилась, если очередь не отсортирована, то сначала добавляем элемент в конец очереди, затем сортируем ее. Если свободных мест нет, то генерируется исключение.

Перегружен оператор вывода в стандартный поток с помощью перегрузки дружественного оператора \textit{operator<<}.
Загрузка структуры данных из стандартного потока реализована с помощью перегрузки дружественного оператора \textit{operator>>}, где происходит считывание элемента массива из потока и его запись в очередь до тех пор, пока очередь не будет заполнена или пока не произойдет ошибка считывания из потока (при необходимости пользователь может сам прервать заполнение очереди, нажав, например, букву с клавиатуры). При этом для элементов очереди должны быть переопределены операции \textit{<<} и \textit{>>}.

Поскольку очередь циклическая, то дополнительно были определены методы \textit{toIndex}, \textit{incHead} и \textit{incTail}, которые вычисляют циклические значения переменных.

\subsection{Процедура получения исполняемых программных модулей}
Программный код был скомпилирован с среде \textit{Visual Studio 2017}. Компиляция раздельная. Код программы содержится в разных файлах. Никаких дополнительных ключей не добавлялось, использовались ключи, которые добавляются по умолчанию.
\subsection{Результаты тестирования}
Для тестирования программы был применен механизм UnitTest, все тесты были пройдены успешно.
Также в функции \textit{main} файла Main.cpp представлено дополнительное тестирование.
\newpage
\setcounter{figure}{1} 
\setcounter{section}{1} 
\setcounter{subsection}{1} 

\begin{center}
	\section*{Приложение А}
	полный код программы
	\addcontentsline{toc}{section}{Приложение А}
	
\end{center}

\renewcommand{\subsection}{\Asbuk{section}.\arabic{subsection}}
\setcounter{subsection}{1} 
\textbf{\subsection{  - Main.cpp}}
\addcontentsline{toc}{subsection}{Main.cpp}
\lstinputlisting[language=C++]{../CircleQueue/Main.cpp}

\setcounter{subsection}{2} 
\textbf{\subsection{  - BigInt.h}}
\addcontentsline{toc}{subsection}{BigInt.h}
\lstinputlisting[language=C++]{../CircleQueue/CircleQueue.h}

\setcounter{subsection}{3} 
\textbf{\subsection{  - Unittest1.cpp}}
\addcontentsline{toc}{subsection}{Unittest1.cpp}
\lstinputlisting[language=C++]{../CicrleQueueTest/unittest1.cpp}


\end{document} % конец документа
