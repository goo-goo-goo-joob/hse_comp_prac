%!TEX TS-program = xelatex

% Шаблон документа LaTeX создан в 2018 году
% Алексеем Подчезерцевым
% В качестве исходных использованы шаблоны
% 	Данилом Фёдоровых (danil@fedorovykh.ru) 
%		https://www.writelatex.com/coursera/latex/5.2.2
%	LaTeX-шаблон для русской кандидатской диссертации и её автореферата.
%		https://github.com/AndreyAkinshin/Russian-Phd-LaTeX-Dissertation-Template

\documentclass[a4paper,14pt]{article}

\input{data/preambular.tex}
\begin{document} % конец преамбулы, начало документа
\input{data/title.tex}
\setcounter{page}{2} % нумерация

\renewcommand\contentsname{\centering {\normalsize Содержание}}
\tableofcontents
\newpage

\section*{Постановка задачи}
\addcontentsline{toc}{section}{Постановка задачи}

Сделать разработанный в рамках лабораторных работ №1 и №2 тип данных производным от класса «ADT», переопределив в нем соответствующие методы. Выполнить аналогичную процедуру еще над
каким-либо простым классом (например, даты или целого числа). Переделать шаблон структуры данных из перечня, приведенного в задании лабораторной работы №3, в класс, хранящий указатели на
объекты класса «ADT». Разработать программу, демонстрирующую возможность хранения в одной структуре данных объектов различного типа. Все задания выполнить в соответствии с вариантом 4.

\newpage

\section{Основная часть}
\subsection{Общая идея решения задачи}
Для решения задачи был разработан базовый класс \textit{ADT} со всеми необходимыми виртуальными методами, позволяющими описывать общий интерфейс для работы с разными объектами из задачи. Было применено наследование. Показана возможность хранить разнородные объекты в одной структуре данных с помощью общего интерфейса объектов. 
\subsection{Структура и принципы действия}
В лабораторной работе №4 был разработан абстрактный базовый класс, содержащий следующие методы: перегруженные операторы сравнения \textit{==, !=, <, <=, >, >=}, дружественные функции ввода и вывода в поток, функция \textit{GetKind}, позволяющая определить конкретный тип класса, вспомогательные функции \textit{print} и \textit{scan}, позволяющие осуществлять ввод и вывод.

Далее класс \textit{BigInt} из лабораторных работ 1 и 2 был сделан производным от класса \textit{ADT}. В нем были переопределены все виртуальные функции базового класса. Дополнительно была применена функция \textit{toDerived}, которая приводила \textit{ADT} к типу \textit{BigInt}.

В рамках задачи был реализован класс даты \textit{Date}, наследник класса \textit{ADT}. Класс содержит поля, которые хранят день, месяц и год. Класс содержит конструктор по умолчанию и конструктор с параметрами. В классе переопределены все виртуальные функции базового класса. Аналогично классу \textit{BigInt} было применено проведение типов.

Далее шаблонный класс \textit{CircleQueue} был заменен на класс, хранящий указатели на объекты структуры \textit{ADT}. Данный пример показывает, как в одной такой очереди можно хранить и экземпляры классов \textit{BigInt}, и классов \textit{Date}.

\subsection{Процедура получения исполняемых программных модулей}
Программный код был скомпилирован с среде \textit{Visual Studio 2017}. Компиляция раздельная. Код программы содержится в разных файлах. Никаких дополнительных ключей не добавлялось, использовались ключи, которые добавляются по умолчанию.
\subsection{Результаты тестирования}
Для тестирования программы был применен механизм UnitTest, все тесты были пройдены успешно.
Для лабораторных работ 1, 2 и 3 были применены тесты, написанные ранее. В тестирование очереди из лабораторной работы 3 были включены  классы \textit{BigInt} и \textit{Date}. Дополнительно было применено тестирование класса \textit{Date}.
\newpage
\setcounter{figure}{1} 
\setcounter{section}{1} 
\setcounter{subsection}{1} 

\begin{center}
	\section*{Приложение А}
	полный код программы
	\addcontentsline{toc}{section}{Приложение А}
	
\end{center}

\renewcommand{\subsection}{\Asbuk{section}.\arabic{subsection}}
\setcounter{subsection}{1} 
\textbf{\subsection{  - BigInt.h}}
\addcontentsline{toc}{subsection}{BigInt.h}
\lstinputlisting[language=C++]{../../BigInt/BigInt/BigInt.h}

\setcounter{subsection}{2} 
\textbf{\subsection{  - CircleQueue.h}}
\addcontentsline{toc}{subsection}{CircleQueue.h}
\lstinputlisting[language=C++]{../CircleQueue/CircleQueue.h}

\setcounter{subsection}{3} 
\textbf{\subsection{  - ADT.h}}
\addcontentsline{toc}{subsection}{ADT.h}
\lstinputlisting[language=C++]{../../ADT/ADT/ADT.h}

\setcounter{subsection}{4} 
\textbf{\subsection{  - ADT\_types.h}}
\addcontentsline{toc}{subsection}{ADT\_types.h}
\lstinputlisting[language=C++]{../../ADT/ADT/ADT_types.h}

\setcounter{subsection}{5} 
\textbf{\subsection{  - Date.h}}
\addcontentsline{toc}{subsection}{Date.h}
\lstinputlisting[language=C++]{../../ADT/ADT/Date.h}

\setcounter{subsection}{6} 
\textbf{\subsection{  - BigInt - unittest1.cpp}}
\addcontentsline{toc}{subsection}{BigInt - unittest1.cpp}
\lstinputlisting[language=C++]{../../BigInt/UnitTest1/unittest1.cpp}

\setcounter{subsection}{7} 
\textbf{\subsection{  - CicrleQueue - Unittest1.cpp}}
\addcontentsline{toc}{subsection}{CicrleQueue - Unittest1.cpp}
\lstinputlisting[language=C++]{../CicrleQueueTest/unittest1.cpp}

\setcounter{subsection}{8} 
\textbf{\subsection{  - ADT - unittest1.cpp}}
\addcontentsline{toc}{subsection}{ADT - unittest1.cpp}
\lstinputlisting[language=C++]{../../ADT/ADTTest/unittest1.cpp}


\end{document} % конец документа
