%!TEX TS-program = xelatex

% Шаблон документа LaTeX создан в 2018 году
% Алексеем Подчезерцевым
% В качестве исходных использованы шаблоны
% 	Данилом Фёдоровых (danil@fedorovykh.ru) 
%		https://www.writelatex.com/coursera/latex/5.2.2
%	LaTeX-шаблон для русской кандидатской диссертации и её автореферата.
%		https://github.com/AndreyAkinshin/Russian-Phd-LaTeX-Dissertation-Template

\documentclass[a4paper,14pt]{article}

\input{data/preambular.tex}
\begin{document} % конец преамбулы, начало документа
\input{data/title.tex}
\setcounter{page}{2} % нумерация

\renewcommand\contentsname{\centering {\normalsize Содержание}}
\tableofcontents
\newpage

\section*{Постановка задачи}
\addcontentsline{toc}{section}{Постановка задачи}

Разработать класс, объект которого реализует тип данных, указанный в варианте. Обеспечить его произвольную размерность за счет использования в объекте динамических структур данных. Разработать
необходимые конструкторы, деструктор, конструктор копирования, а также методы, обеспечивающие
изменение отдельных составных частей объекта (например, коэффициентов полинома) и вывод его содержимого. Выполнить задания в соответствии с вариантом 4: целое произвольной длины во внешней форме представления в виде строки символов-цифр.

\newpage

\section{Основная часть}
\subsection{Общая идея решения задачи}
Для решения задачи были использованы:
\begin{enumerate}
	\item Aлгоритмы стандартных библиотек \textit{<algorithm>}: \textit{max}, \textit{<sting>}: \textit{push\_back, reverse}.
	\item Контейнер \textit{<string>}.
\end{enumerate}
Для обеспечения произвольной размерности использовался массив из элементов типа \textit{char}. Был разработан класс \textit{BigInt} со всеми необходимыми методами. 
\subsection{Структура и принципы действия}
Для выполнения задания был создан класс \textit{BigInt}, содержащий закрытые поля типа \textit{char* value} - хранит одну символ-цифру числа без учета знака в одном элементе массива (для хранения используется десятичная система счисления), где единичный разряд - нулевой элемент массива, \textit{size\_t size} - хранит количество цифр в числе, \textit{bool sign} - хранит знак числа.

Класс содержит конструктор с параметром (принимает строку): если в начале строки стоит знак минус, то \textit{sign} примет значение \textit{false}, инчает \textit{true}. Также происходит проверка на то, что в строке все элементы цифры, в противном случае выбрасывается исключение. Конструктор инициализирует длину и заполняет массив элементами строки. Класс также содержит конструктор копирования и деструктор. Также реализована возможность вывода числа в поток с помощью перегрузки оператора \textit{operator<<}.

Для выполнения задачи мне потребовались следующие методы: аксессоры к закрытым полям (\textit{getValue}, \textit{getSize}, \textit{getSign}), возможность изменения знака числа (\textit{setSign}, \textit{operator-}), возможность сравнения чисел (\textit{operator<}, \textit{operator==}). С помощью них были реализованы методы, обеспечивающие изменение отдельных составных частей объекта (в данном случае изменение самого числа и его знака) - методы сложения и вычитания (перегрузка операторов \textit{operator+},  \textit{operator-}, \textit{operator+=}, \textit{operator-=}, копирующий оператор присваивания). 

Сложение было реализовано следующим образом: при равенстве знаков у чисел происходило посимвольное сложение, начиная с младшего разряда, также отдельная переменная учитывала перенос десятков к следующему разрядку. Если оба числа были отрицательными, то на результат навешивался знак минус. Если числа были разных знаков, то результат сводился к операции разности.

Для операции разности была использована возможность сравнения чисел. Если оба числа положительные и уменьшаемое больше вычитаемого, то происходило посимвольное вычитание с возможностью занятия десятка у старшего разряда, иначе менялся порядок вычитания и у результата менялся знак на минус. При наличии разных знаков у чисел результат сводился к сложению. Если оба числа отрицательные, то меняли порядок разности, изменяя знаки чисел соответствующим образом.

\subsection{Процедура получения исполняемых программных модулей}
Программный код был скомпилирован с среде \textit{Visual Studio 2017}. Компиляция раздельная. Код программы содержится в разных файлах. Никаких дополнительных ключей не добавлялось, использовались ключи, которые добавляются по умолчанию. Версия пакета \textit{SDK} для \textit{Windows}: 10.0.17763.0
\subsection{Результаты тестирования}
Для тестирования программы был применен механизм UnitTest, все тесты были пройдены успешно.
Также в функции main файла Main.cpp представлено дополнительное тестирование. Ожидаемый вывод функции: 
\begin{verbatim}
-1267650600228229401496703205376
-1267650600228229401496703205376

128667160846894790715675442493073
1
\end{verbatim}

\newpage
\setcounter{figure}{1} 
\setcounter{section}{1} 
\setcounter{subsection}{1} 

\begin{center}
	\section*{Приложение А}
	полный код программы
	\addcontentsline{toc}{section}{Приложение А}
	
\end{center}

\renewcommand{\subsection}{\Asbuk{section}.\arabic{subsection}}
\setcounter{subsection}{1} 
\textbf{\subsection{  - Main.cpp}}
\addcontentsline{toc}{subsection}{Main.cpp}
\lstinputlisting[language=C++]{../BigInt/Main.cpp}

\setcounter{subsection}{2} 
\textbf{\subsection{  - BigInt.h}}
\addcontentsline{toc}{subsection}{BigInt.h}
\lstinputlisting[language=C++]{../BigInt/BigInt.h}

\setcounter{subsection}{3} 
\textbf{\subsection{  - Unittest1.cpp}}
\addcontentsline{toc}{subsection}{Unittest1.cpp}
\lstinputlisting[language=C++]{../UnitTest1/unittest1.cpp}


\end{document} % конец документа
